\pdfoutput=1

\documentclass[11pt]{article}

\usepackage[]{ACL2023}

\usepackage{times}
\usepackage{latexsym}

\usepackage[T1]{fontenc}

\usepackage[utf8]{inputenc}

\usepackage{microtype}

\usepackage{inconsolata}

\title{Asking Clarifying Questions for Conversational Search}

\author{
  Ansh Bisht \and
  Jordan Dickson \and
  Jack Harrison \and
  Ruben Lazell \and
  Andrew Taison  \\
  School of Computing, Engineering and the Built Environment, \\
  Edinburgh Napier University \\
  Matric Numbers: 40527530, 40545300, 40537035, 40679914, 40538519
}


\begin{document}
\maketitle
\begin{abstract}
Short abstract.
\end{abstract}


\section{Introduction}
Brief context of problem - why clarifying Q's matter.
What is the goal.
Structure of the paper.

\section{Related Work}
Existing systems, prior research etc.

\section{Methodology}
System architecture.
What models/algorithms used.
How are the questions generated.
What inputs/outputs are there.

\cite{Aliannejadi2019}

\section{Evaluation}
How did we check the system works.
Describe testing process.

\section{Results and Discussion}
Sample outputs.
Summary of system behaviour.
What worked well, what didn't?
Possible improvements.

\section{Conclusion}
start here.

\section*{Limitations}
Required by ACL format, and should be AFTER conclusion.
Discuss honest limitations of the work.

\section*{Ethics Statement}
Required by ACL format. Could just be a sentence or two.
Explicit ethics statement on the broader impact of the work, or other ethical considerations.

\bibliographystyle{acl_natbib}
\bibliography{references}

\appendix

\section{Appendix}
Possibly not needed.

\end{document}